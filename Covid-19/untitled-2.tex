% build by DOING following: run latex, bibtex, latex, latex
%%%%%%%%%%%%%%%%%%%%%%%%%%%%%%%%%%%%%%%%%%%%%%%%%%%%%%%%%%%%%

%------------------------------------------------------------
%
\documentclass{amsart}
%
%----------------------------------------------------------
% This is a sample document for the AMS LaTeX Article Class
%
\usepackage{amsmath}%
\usepackage{amsfonts}%
\usepackage{amssymb}%


\usepackage[dvipdfm]{graphicx}  % to avoid bounding box error of images
\usepackage{bmpsize} % to avoid bounding box error of images

\usepackage{graphicx} % must be placed AFTER "\usepackage[dvipdfm]{graphicx}"

\usepackage{footmisc} % to reuse footnotes

\usepackage{hyperref} %szy: to create clickable links for citings
\usepackage{color}

%szy: following used for creating flow diagram:
\usepackage{tikz}
\usetikzlibrary{arrows}

\usepackage[normalem]{ulem} %szy: for text strikethroughs
\usepackage[mathscr]{euscript}

\usepackage{caption}
\usepackage{subcaption}
\usepackage{float}

\usepackage[margin=1.0in]{geometry} %set custom margins

\makeatletter
\g@addto@macro{\endabstract}{\@setabstract}
\newcommand{\authorfootnotes}{\renewcommand\thefootnote{\@fnsymbol\c@footnote}}%
\makeatother

\usepackage{fancyhdr}
\pagestyle{fancy}
\thispagestyle{empty}
\fancyhead[CE]{\begin{footnotesize}\uppercase{COVID-19}\end{footnotesize}}
\fancyhead[CO]{\begin{footnotesize}\uppercase{S. Szymanowski}\end{footnotesize}}
\renewcommand{\headrulewidth}{0pt}


%--------------------------------------------------------
\begin{document}


\begin{center}

\begin{large} % serving as \title:
\uppercase{\textbf{COVID-19 IN THE UNITED STATES}}
\end{large}

\vspace{5mm}

\normalsize
\authorfootnotes
\textsc{
Rasheed Hameed
}
%\textsc{
%Steve Szymanowski\textsuperscript{1}, Rasheed Hameed\textsuperscript{1}, Ridouan Bani\textsuperscript{2},
%\\ Paul J. Gruenewald\textsuperscript{3}, Carlos Castillo-Ch\'{a}vez\textsuperscript{4}, and Anuj Mubayi\textsuperscript{1,4}
%}

\vspace{3mm}

\begin{small}
%\textsuperscript{1}DePaul University \\
\vspace{1mm}
%\textsuperscript{2}McGill University \\
%845 Sherbrooke Street West, Montreal, Quebec, Canada H3A 0G4 \\
%\vspace{1mm}
%\textsuperscript{3}Prevention Research Center \\
%180 Grand Ave, Oakland, CA 94612 \\
%\vspace{1mm}
%\textsuperscript{4}Simon A. Levin Mathematical, Computational, and Modeling Sciences Center \\
%Arizona State University \\
%P.O. Box 873901, Tempe, AZ, 85287-3901
\end{small}
\end{center}


%%%%%%%%%%%%%%%%%%%%%%%%%%%%%%%%%%%%%%%%%%%%%%%%%%%%%%%%%%%%%%%%%%%%%%%%%%%%%%%%%%%%%%%%%%

\par
The covid-19 model is represented by a system of ordinary differential equations shown here~\ref{eq:covid19system}.
\begin{align}\label{eq:covid19system} 
	\frac{dG}{dt}& = \Lambda - \beta_1 \frac{N}{O}G - \mu G \\
	\frac{dS}{dt}& = \beta_1 \frac{N}{O} G
					 - \beta_2 \frac{(E+kC)}{N} S
					 - \mu S \notag \\
	\frac{dE}{dt}& = \beta_2 \frac{(E+kC)}{N} S
					- \gamma E
					- \mu E \notag \\
	\frac{dC}{dt}& = \gamma E
 					-\theta_C C
					- \mu C \notag \\
	%\frac{dA}{dt}& = 	 \notag \\
	\frac{dR}{dt}& = 	 \theta_C C
					- \mu R \notag
\end{align}
where:
\begin{align*}
	N& = S+E+C+R \\
	O& = G + N
\end{align*}
\par

%%%%%%%%%%%%%%%%%%%%%%%%%%%%%%%%%%%%%%%%%%%%%%%%%%%%%%%%%%%%%%%%%%%%%%%%%%%%%%%%%%%%%%%%%%

%%%%%%%%%%%%%%%%%%%%%%%%%%%%%%%%%%%%%%%%%%%%%%%%%%%%%%%%%%%%%%%%%%%%%%%%%%%%%%%%%%%%%%%%%%%%%%%%%%%%%%%%
%\tikzstyle{int}=[draw, minimum size=2em]
\tikzstyle{int}=[circle, draw, minimum size=3em]
%\tikzstyle{init} = [pin edge={to-,thin,black}]
\tikzset{force/.style={text width=1.5cm, text badly centered, minimum height=1cm, font=\footnotesize}}

%%%\begin{figure}[h!]
\begin{figure}[H]
	\begin{center}
	\begin{tikzpicture}[->,.=stealth',shorten >=1pt,auto,node distance=2.75cm,auto,>=latex']
    	\node [coordinate] [node distance=0.25cm] (start) {}; %szy: used as blank, starting node, to point from
		\node [int] (G) [right of=start] {$G$};
			%\node [coordinate] (endpointG1) [right of=G] {}; %szy: blank, ending node to point to
			\node [coordinate] (endpointG) [below of=G] {}; %szy: blank, ending node to point to


		\node [int] (S) [right of=G] {$S$};
			\node [coordinate] (endpointS) [below of=S] {};
    	
    	\node [int] (E) [right of=S] {$E$};
    				\node [coordinate] (endpointE) [below of=E] {};
    				
%		\node [int] (C) [above of=E] {$C$};
    %				\node [coordinate] (endpointC) [right of=C] {};

		\node [int] (C) [right of=E] {$C$};
    				\node [coordinate] (endpointC) [below of=C] {};
    				

    	%\node [int] (R) [right of=E] {$R$};
    	%			\node [coordinate] (endpointR) [below of=R] {};    	
    	    	\node [int] (R) [right of=C] {$R$};
    				\node [coordinate] (endpointR) [below of=R] {};    	
		
	%%%%%%%%%%%%%%%%%%%%%%%%%%%%%%%%%%%%%%%%%%%%%%%%%%%%%%%%%%%%%%%%%%%%%
		
		%\path[every node/.style={font=\sffamily\small}]
		\path[every node]
				[->] (start) edge node {$\Lambda$} (G)
				[->] (G) edge node[near end] {$\beta_1 \frac{N}{O}G$} (S)
				[->] (G) edge node[] {$\mu G$} (endpointG)
				
				[->] (S) edge node {$\beta_2 \frac{(E+kH)}{N} S$} (E)
				[->] (S) edge node[] {$\mu S$} (endpointS)
				

				[->] (E) edge node {$\gamma E$} (C)
				[->] (E) edge node[left] {$\mu E$} (endpointE)

				[->] (C) edge node[] {$\theta_C C$} (R)
				[->] (C) edge node[] {$\mu C$} (endpointC)


				[->] (R) edge node[] {$\mu R$} (endpointR)
				;


		\node [force] (label1) at (2.80,3.75) {General\\Population}; % [force] forces the "\\"
		\node [force] (label2) at (5.00,3.75) {At-Risk\\Population};

		\draw[-,dashed] (3.90,4.35) -- (3.90,-4.50);  
						
		\end{tikzpicture}
	\end{center}
	\caption{COVID-19 Model}
	\label{model_diagram}
\end{figure}



\begin{minipage}{1.0\textwidth}
	\begin{table}[H]
		\centering
		\scalebox{0.8} { % use scale to size table
		\begin{tabular}{|c|l|c|c|}
			\hline

			\textbf{Variables} & \textbf{Description} & \textbf{Initial Value}\footnote{total count} & \textbf{Source} \\
		  	\hline
			$G$ & General Population of US & 330,511,531 & \cite{CA_pop_1980} \\ \hline
			%%% NOTES on how got initial value for G: data from 1980 (closest to 1983)

			$S$ & susceptible & 79,834,735 & \cite{CA_ACS_poverty} \cite{CA_ACS_edu} \\ \hline
			%%% NOTES on how got initial value for S:
			%%% used 2008-2012 data to get an average rate  \\
			%%% data from citation is ``below poverty: 11.5\%'' \cite{CA_ACS_poverty}\\
			%%% data from citation is ``education attainment \ldots : 10.3\% + 8.7\% = 19.0\%''  \cite{CA_ACS_edu}\\
			%%% $[(0.115 + 0.19)/  2 =  0.1525]*23667902=3609400$

			$E$ & Expected population of individuals infected with virus & 20,000 & derived \\ \hline
			%%% NOTES on how got initial value for U:
			%%% derived from: b/c I assume 6600/0.33 and 0.33 of U goto H \\

			$C$ & Confirmed cases of Coronavirus & 100,000 & derived \\ \hline
			%%% NOTES on how got initial value for H:
			%%% used admissions data for 1983 from data used for fitting \\	


			$R$ & Recovered & 100 & est \\ \hline
			%%% NOTES on how got initial value for R:
			%%% guess--using "H" as bound: $0.25*H(0)=0.25*6600=1650$ \\
	
			$N$ & at-risk population & 79,834,735 & derived \\ \hline
			$O$ & total population & 330,511,531  & derived \\ \hline
			& & & \\ \hline
%%%%%%%%%%%%%%%%%%%%%%%%%%%%%%%%%%%%%%%%%%%%%%%%%%%%%%%%%%%%%%%%%%%%%%%%%%%%%%%%%%%%%%%%%%%%%%%%%%%%%%

			%\textbf{Parameter} & \textbf{Description} & \textbf{Value \footnote{1 time unit = 1 year}} & \textbf{Source} \\
			\textbf{Parameter} & \textbf{Description} & \textbf{Values \& Units \footnote{1 time unit = 1 year}} & \textbf{Source} \\
		  	\hline
			$\Lambda$ & birth rate & 435,722 people per year & \cite{CA_pop_birth_1983} \\ \hline
			$\mu$ & mortality rate & $\mathcal{U}[0.012, 0.013]$ \footnote{parameter sampled from the Uniform distribution in the given range\label{uniform_sample}} per person per year & \cite{CA_avg_life_1981} \\ \hline
	
			
			$\psi$ & release rate & $\mathcal{U}[0.124, 0.164]$ \footref{uniform_sample} per person per year &  \cite{CA_prison_parole} \\ \hline
			$\gamma$ & rehab admissions rate & $\mathcal{E}(0.304)$ \footnote{parameter sampled from the Exponential distribution with the given rate parameter} per person per year  &  \cite{gruenewald2010assessing} \\ \hline
 			$\theta_H$ & quitting rate (due to rehab success) & $0.33$ per person per year &  \cite{CA_FactSheetMeth} \\ \hline
 			$\eta$ & relapse rate & $0.1894$ per person per year &  \cite{brecht2000predictors} \\ \hline
 			$\beta_1$ & growth factors of at-risk population & $0.245$ dimensionless &  \cite{gruenewald2013mapping} \\ \hline
 			$\theta_U$ & quitting rate (on their own) & $\mathcal{U}[0, 0.33]$ \footref{uniform_sample} per person per year &  est \\ \hline
 			$p$ & proportion released that successfully quit & $\mathcal{U}[0.8, 0.9]$ \footref{uniform_sample} dimensionless &  est \\ \hline
 			$k$ & reduction in transmission intensity of H & $\mathcal{U}[0, 1]$ \footref{uniform_sample} dimensionless  & est \\ \hline
 			$\beta_2$ & transmission intensity rate & $\mathcal{N}(\mu=0.75, \sigma=0.47)$ \footnote{parameter sampled from the Normal distribution with the given mean and standard deviation} interactions per person per year & \cite{mubayi2011types} \\ \hline 			
		\end{tabular}
	
	} % end \scalebox

\caption{Brief Summary of State Variables \& Parameters}
\label{table:state_vars_params_brief}

\end{table}

\end{minipage}

%%%%%%%%%%%%%%%%%%%%%%%%%%%%%%%%%%%%%%%%%%%%%%%%%%%%%%%%%%%%%%%%%%%%%%%%%%%%%%%%%%%%%%%%%%%%%%%%%%%%%%


\end{document}